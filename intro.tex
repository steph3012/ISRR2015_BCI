\section{Introduction}

In this paper we cite~\cite{parra}

A necessary condition for realizing the promise of pervasive human-robot collaboration is that the human-robot interface is natural and seamless.  Seamless interaction requires that there are fast feedback channels from the human to the robot and ideally, that there is no need for explicit cueing.  
For example, requiring the human to provide a button press to communicate his/her need to the robot is not only unnatural but it can also be disruptive to the main task at hand (i.e. a collaborative building task where often the human's hands are full).  While robot-parsable language is an improvement over button presses, unfortunately language is often ambiguous.  Ideally, feedback between the robot and the human would be fast (especially important for safety critical tasks), unambiguous, and natural; meaning that the human does not need to modify his/her behavior nor undergo training in order to successfully collaborate with the robot.

The desire for natural feedback between humans and robots has fueled the recent development of methods for using the human operator's electroencephalographic brain-generated signals for communication with robots.  The electrical activity generated by the human brain can be detected non-invasively via an electroencephalography (EEG) sensor placed on the head, and has been shown to consist of mappable patterns that encode 


Communication via the human operator's electroencephalographic signals can  electroencephalogram